\documentclass[12pt,twoside,abbrevs,msc,ai,notimes,logo,sansheadings]{infthesis}

% Packages
\usepackage{graphicx}
\usepackage{abbrevs}
\usepackage{acronym}
% Will use listings instead for now. minted isn't included in the ubuntu texlive version. \usepackage{minted}
\usepackage{multirow}

% Added packages
\usepackage{cite}
\usepackage{listings}
\usepackage{hyperref} % for \url
\usepackage{amstext} % for \text
\usepackage{subcaption} % for \subfloat
\usepackage{epstopdf} % for eps images
\usepackage[T1]{fontenc} % fix encondings?
\usepackage{textcomp}
\usepackage[scaled]{beramono} % for bera mono font
\usepackage[usenames,dvipsnames]{color} % for colour names
\usepackage{comment}

% only needed for \usepackage{fancyvrb}. \renewcommand\theFancyVerbLine{\normalsize\arabic{FancyVerbLine}}

% Added config
\setlength{\parskip}{0pt plus12pt minus4pt}
\raggedbottom

\lstset{
  basicstyle=\ttfamily\scriptsize,
  breaklines=true,
  frame=tb,
  commentstyle={\color{OliveGreen}}
}

% Project Details
\title{Parallelising Plan Recognition}
\author{Chris Swetenham}
\course{MSc Artificial Intelligence}
\project{Dissertation}

\submityear{2012}
\graduationdate{November 2012}
\date{\today}

% TODO other acronyms
\newacro{CAS}{Compare And Swap}
\newacro{CCG}{Combinatorial Categorical Grammar}

\abstract{

Plan Recognition is an area of Artificial Intelligence research with many potential applications in which it is desirable to perform it in real time. In order to devise a performant solution for plan recognition, we study the ELEXIR algorithm for plan recognition in C++. We find it well suited to parallelisation using multithreaded job scheduling techniques and apply several scheduling techniques, including lock-free work stealing. We and analyse and report on the speed and scalability of these techniques when applied to plan recognition.
}


% Extra Package Commands

\begin{document}
  \begin{preliminary}
    % Title Page
    \maketitle

    % Preamble
    \begin{acknowledgements}

	Thanks to my supervisor, Chris Geib, for his support during this project.

\end{acknowledgements}

    \standarddeclaration
    %\dedication{

TODO is this even included

}
 TODO?
    \tableofcontents
    \listoffigures
  \end{preliminary}

% Sides per Section (Planned)
% ---------------------------
% Title page 1
% Blank 1
% Abstract 1
% Acknowledgements 1
% Declaration 1
% Blank 1
% TOC 2
% List of Figures 1
% Blank 1
% Intro 1
% Blank 1
% Background 8
% Related Work 1
% Blank 1
% Initial Work 3
% Blank 1
% Method 0 1
% Blank 1
% Method 1 2
% Method 2 3
% Blank 1
% Method 3 4
% Method 4 2
% Analysis 3
% Blank 1
% Conclusion 1
% Bibliography 2
% --------------
% Total 47

  
  % Chapters
  % Introduction: background to the project, previous work, exposition of relevant literature, setting of the work in the proper context.
  \chapter{Introduction}

TODO one-page intro
  \chapter{Background}

In this section, we discuss the background for the project and provide context for the work performed.

\section{Plan Recognition}

Plan Recognition looks at a sequence of actions by an agent, and attempts to deduce that agent's goals. There are many possible situations where it is desirable to recognise the intentions of an agent from its observed actions. In network intrusion detection, we wish to identify any actions which could correspond to an attack, hidden in a large number of benign actions by users, in real time\cite{bib:netsec}. In human-robot interaction, we wish a human and a robot to work together on a task. If the robot partner can observe the actions of the human partner, and from them understand what the human partner is currently trying to achieve, it an anticipate the needs of the human partner, and assist or offer advice\cite{bib:jast}. In assisted care scenarios, a robot can again anticipate the needs of the human being assisted; it can also recognise unusual behaviour which would necessitate emergency response\cite{bib:assistive}.

Since Plan Recognition takes symbolic input, some amount of preprocessing of the input may be required. For network intrusion detection, this could be a simple filter on system logs. For more complex scenarios, Plan Recognition could tie in to a system which takes video and other real-time feeds and classifies the actions taken by humans using machine learning techniques.

Some Plan Recognition techniques use a formal grammar to define actions and plans to be recognised. In this project, we use two grammars crafted by hand to fit different problem domains. The formal grammars are defined in the language of \emph{Combinatory Categorial Grammars}, discussed below.

\section{Combinatory Categorial Grammars}

The problem of plan recognition is very similar to the problem of parsing, where we analyse a sequence of tokens and produce a syntax tree according to a formal grammar; so it is natural to consider the applicability of parsing techniques to this problem. When parsing computer languages such as XML or C++, ambiguity is undesirable. An input text should admit to either a single interpretation or no interpretation at all, and it is the task of the language designer to resolve any possible ambiguities in the grammar.

When parsing a sequence of actions into possible explanations, a degree of ambiguity is unavoidable; for example, if we observe a truck driving from Edinburgh to Newcastle, it may be making a delivery locally in Newcastle; or it may be driving there in its way to York. Grammars and parsers for natural language must deal with potential ambiguity and multiple possible parses in human language. Therefore, we can expect parsing techniques from natural language processing may be more amenable to being adapted to the problem of Plan Recognition than parsing techniques for computer languages.

Categorial Grammars\cite{bib:ccg} assign \emph{categories} to each input token or symbol, and combine them according to simple rules. \emph{Basic} categories correspond to lexical classes such as ``identifier'' in a computer language or ``noun'' in a natural language. We denote basic categories by atomic terms such as $A$. \emph{Complex} categories are akin to functions over categories: they look for arguments to their left (denoted $\backslash$) or right (denoted $/$) in the sequence of categories, and yield other categories as a result. A complex category taking an argument $A$ to its left and yielding $B$ is denoted $B\backslash A$; one taking the same argument to the right is denoted $B/A$. \footnote{Some sources denote the leftward example above as $A\backslash B$, preserving the ordering of the argument and result. For consistency we will always denote the arguments on the right and the results on the left.}

Taking a simple example from the Logistics domain:

\begin{lstlisting}
observations: [
  loadTruck(p0, t0, c1),
  driveTruck(t0, c1, c0),
  unloadTruck(p0, t0, c0)
];
\end{lstlisting}

We have three observations: package \texttt{p0} is loaded into truck \texttt{t0} at location \texttt{c1}, the truck drives to location \texttt{c0}, and the package is unloaded there. We then assign basic or complex categories to each observation:

\begin{lstlisting}
loadTruck(p0, t0, c1) :
  inTruckC(p0, t0)
driveTruck(t0, c1, c0) :
  (deliverPkgC(o, c0)/notInTruckC(o, t0))\inTruckC(obj, t0)o
unloadTruck(p0, t0, c0) :
  notInTruckC(p0, t0)
\end{lstlisting}


We can then combine the complex categories with their arguments and produce a parse. Combining the categories follows the rules: $ X\leftarrow X/Y,\; Y$ and $ X\leftarrow Y,\; X\backslash Y$. Note that package \texttt{o} is unbound until the complex category is unified with an argument.

\noindent\rule{\textwidth}{0.4pt}
{\ttfamily\scriptsize
$\underbrace{\text{inTruckC(p0, t0)},\quad \text{(deliverPkgC(o, c0)}/\text{notInTruckC(obj, t0))}\backslash{}\text{inTruckC(o, t0)}},\quad \text{notInTruckC(p0, t0)}$\\
$\hspace*{10em}\underbrace{\text{deliverPkgC(p0, c0)}/\text{notInTruckC(p0, t0)},\hspace*{9em}\text{notInTruckC(p0, t0)}}$\\
$\hspace*{24em} \text{deliverPkgC(p0, c0)}$
}\\
\noindent\rule{\textwidth}{0.4pt}

We reduce the input to an explanation consisting of a single basic category, \texttt{deliverPkgC(p0, c0)}.
\emph{Combinatory Categorial Grammars}\cite{bib:ccg} (CCGs) extend categorial grammars with new reductions corresponding to combinators in combinatory logic. Since complex categories can take arguments either to their left or right, each combinator has a leftward and a rightward version.

The leftward and rightward application combinators are the reductions already present in categorial grammars. The other two combinators commonly used in CCGs are composition and type-raising.

Composition corresponds to function composition. The rightward composition operation has the form: $X/Z \leftarrow X/Y,\;  Y/Z$. Type-raising transforms basic categories to complex categories. The rightward type-raising operation has the form: $T/(T\backslash X) \leftarrow X$.

\section{Applying CCGs to Plan Recognition}

In applying CCGs to Plan Recognition, we seek to assign (basic or complex) categories to the input sequence of actions, and by reductions using a set of combinators, produce a set of explanations for the input. Adjacency of grammar elements is not as crucial in Plan Recognition as in Natural Language Processing, and so we make some modifications to the CCG formalism to apply it to Plan Recognition.
Firstly, it is possible for intervening actions to occur which are not part of the current goal; they may be part of some other goal, or entirely irrelevant to the goals we are interested in. We therefore allow complex categories to find their arguments anywhere to the left (for leftward-applying categories) or to the right (for rightward-applying categories), rather than directly to the left or right.

Secondly, certain actions to achieve a goal may in certain cases be executed in any order with equal results. Although this could be expressed in the CCG formalism, by including every possible permutation in the grammar, it is preferable to allow complex categories to take sets of arguments which may be provided in any order. This is denoted as $A/\{B, C\}$ for a complex category taking arguments $B$ and $C$ to the right in any order, and yielding $A$\cite{bib:ccg2}.

\section{The ELEXIR Algorithm}

The ELEXIR algorithm\cite{bib:elexir} provides several additional features to CCGs used for plan recognition. \emph{Features} give first-order variable arguments to basic categories, which refer to labels in the domain. For example, \texttt{putAwayC(cup1)}. This allows categories in the domain to refer to many different potential objects, rather than have to repeat similar categories for each potential object in the domain. The same feature label refers to the same object in all basic categories within a complex category.
When categories are combined with combinators, categories which were previously tested for equality must now be \emph{unified}. When we unify two categories, if the same feature is bound in both of them, it must be bound to the same value; otherwise unification fails. If the same feature is bound in one of them and unbound in the other, the unbound feature takes the value of the bound feature. \emph{States}, defined in terms of first-order predicates, are tracked through each explanation, 
each 
action updating the current state predicates. Assignments of categories to each action can be made conditional on certain 
predicates holding in the current state. \emph{Probabilities} for each explanation are computed, based on probabilities assigned in the input to categories and assignments of categories to actions. These probabilities can be made conditional on the current state predicates.

The ELEXIR algorithm restricts itself to three combinators: left application, right application, and right composition. It builds up a list of explanations by successively adding categories for new observations to existing explanations, and then applying any applicable combinators to pairs of categories in the explanation.
Each input action can modify the state predicates, and can be assigned one or more categories, depending on whether preconditions on their applicability are satisfied by the current state. Combinators are applicable if the relevant arguments or results can be unified. When a new category is added for an observed action, at least one new explanation is produced for each existing one: the one in which no combinators are applied.

In the ELEXIR algorithm, each existing explanation is processed independently when a new action is added to the observations. For each existing explanation, the algorithm tries to apply the three combinators mentioned above between pairs of categories. Each successful application creates a new explanation; the case where no combinators are applied is also a valid new explanation. This process does not refer to any other existing explanations, and does not make modifications to the problem state outside the set of current explanations. This makes ELEXIR well suited to parallelising using task scheduling: depending on the overhead of the method we are using, we can create tasks containing some small or large number of explanations, and schedule these tasks to be executed independently across multiple threads. We now look at techniques for parallelism and how we might apply them to speed up this algorithm.

\section{Concurrency Techniques}

We can split our discussion of concurrency techniques between blocking and non-blocking methods. Blocking methods include those involving mutual exclusion locks (mutexes), semaphores, condition variables, and other constructs where a thread may block on a concurrency construct indefinitely. Mutexes are used to guard access to an object. Before accessing the object, threads must lock the mutex. Only one thread at a time is permitted to hold the mutex locked; other threads which attempt this will sleep until the lock becomes available.
This mechanism allows an object designed to be accessed in a single-thread manner, be safely accessed by multiple threads. Condition variables provide a way for threads to wait until a condition is satified by a shared object, for example for a queue to become non-empty. Threads which modify the object in such a way that the condition might now hold where it previously did not, should notify the condition variable, which will then wake the sleeping threads. These threads will 
then test whether the condition does indeed hold. Condition variables are often used in conjunction with mutexes, to guard access to the object being waited on.

Non-blocking methods offer several possible levels of guarantee. \emph{Wait-free} algorithms guarantee that threads will always complete in a finite number of steps, but are usually expensive. \emph{Lock-free} algorithms only guarantee that some thread will always be making progress, and can be much cheaper than wait-free algorithms. Non-blocking algorithms depend on lower-level primitives than blocking ones; most commonly, the \emph{Compare-And-Swap} (CAS) operation available on modern processors along with memory fences which provide some guarantees with respect to the ordering of memory accesses. Non-blocking algorithms are harder to implement and reason about than blocking methods, but can provide improved performance and lower overhead in return.

There are many potential performance issues in multi-threaded code. The two most relevant to this project are \emph{lock contention}, where threads spend a lot of time waiting on the same mutex for access to a shared object; and \emph{cache contention}, where threads concurrently access memory in the same cache line, causing delays as it is acquired by the caches of the processors the threads are executing on. Results from cachegrind allow us to discover potential cases of cache contention.

\section{Multicore Task Scheduling Techniques} \label{sec:sched}

Many problems can be broken down into individual units of work to be executed across several threads\cite{bib:aomp}. These tasks may have dependencies on other tasks which they require to be complete before they can begin; and they may generate new tasks to be scheduled. There are many possible techniques for scheduling work across threads, with different overheads and complexities of implementation. The lower the overhead of scheduling, the more fine-grained we can make the tasks, and the better we can load-balance to ensure all threads perform useful work. We will discuss several possible techniques for job scheduling on multicore machines, which we will evaluate in parallelising the ELEXIR algorithm.

\begin{description}
\item[Global Queue] One of the simplest possible implementations uses a single global queue for all tasks. Access to the queue is guarded by a global lock. Tasks are enqueued at the tail of the queue, and dequeued at the head. This implementation guarantees that a thread will be able to find a task if one is available, but the contention by all threads on a single global lock means that the overhead of this method is potentially high especially for short-lived tasks.

\item[Blocking Queues] A potentially better implementation uses one queue of work for each worker thread. This reduces contention when threads are requesting work from the queues. In order to ensure that work is distributed to all threads, the main thread must periodically distribute work to all the threads, and worker threads must feed any new tasks to be scheduled back to the main thread. While this reduces contention, worker threads may not be able to find tasks to execute if they are available but the main thread has not scheduled them yet.

\item[Work-Stealing Queues] The final case we will examine is a lock-free work-stealing scheduler, with one queue for each worker thread. In a work-stealing scheduler, threads which find their own queue empty will select a random \emph{victim} thread and attempt to steal a task from their queue. New tasks created by a thread will be added to the thread's own queue. In this way, work will be distributed between threads, but most of the accesses will be uncontended accesses to a thread's own queue. This can be implemented lock-free and has potentially the lowest contention of the methods mentioned here while balancing work well across threads.
\end{description}

  \chapter{Related Work}

[TODO: rewrite]

Approaches to Plan Recognition based on Hidden Markov Models (HMMs) and Conditional Random Fields (CRFs) exist[TODO: cite], but these suffer from contextual issues as well and are better suited to lower-level activity recognition. Graph-based methods reduce plan recognition to a graph covering problem, but are not well able to deal with partially ordered or interleaved plans, contextual effects, or combinations of planning and plan recognition[TODO: cite].
The literature on parallelising search problems is extensive[TODO: cite], but of little relevance to this project since we are performing an exhaustive rather than heuristic search. There is some work on parallelising parsing, but this applies mainly to traditional parsing problems where only a single parse will result[TODO: cite].
Clark and Curran have worked on a parallel implementation for learning CCGs in Natural Language[TODO: cite], but the parsing is not a complete parallel parser like the one we develop here.

ELEXIR is an algorithm for parsing Combinatorial Categorical Grammars for Plan Recognition[TODO: cite]. The current implementation of the ELEXIR algorithm in C++ was well suited to parallelisation. A parallel implementation of the ELEXIR algorithm should show significant speedup on modern multicore machines.

[TODO: cite] provides an overview of literature on work-stealing vs work-sharing as a scheduling strategy. 

[TODO: cite] Chase \& Lev describe an algorithm for a work-stealing queue, which we base ourselves on for our own implementation.

\begin{comment}
[1] Goldman, R. P.; Geib, C. W.; and Miller, C. A. 1999.
plan recognition. In
Intelligence.
A new model of
Proceedings of the Conference on Uncertainty in Artificial
[2] Geib, C. W. 2009.
Delaying Commitment in Plan Recognition Using
Proceedings of the International Joint
Conference on Artificial Intelligence (IJCAI) 2009, 1702-1707.
Combinatory Categorial Grammars. In
[3] Blumofe, R.D.; Leiserson, C.E. 1994. Scheduling multithreaded computa-
Foundations of Computer Science, 1994 Proceedings.,
35th Annual Symposium on, 356-368.
tions by work stealing. In
[4] Grune, D: Parsing techniques: a practical guide. Springer, 2008. ISBN
978-1-4419-1901-4.
[5] Andrews, G. R. Foundations of Multithreaded, Parallel, and Distributed
Programming. Addison-Wesley, 2000. ISBN 978-0-2013-5752-3.
[6] Clark S.; Curran, J. R. 2003. Log-linear models for wide-coverage CCG
Proceedings of the 2003 conference on Empirical methods in natural
language processing, 97-104.
parsing. In
[7] Avrahami-Zilberbrand, D., and Kaminka, G. A. 2005. Fast and complete
symbolic plan recognition. In
Proceedings IJ-CAI.
[8] Bui, H. H.; Venkatesh, S.; and West, G. 2002.
the Abstract Hidden Markov Model.
Policy recognition in
Journal of Artificial Intelligence Research
17:451-499.
\end{comment}
  % Description of the work undertaken: this may be divided into chapters describing the conceptual design work and the actual implementation separately. Any problems or difficulties and the suggested solutions should be mentioned. Alternative solutions and their evaluation should be included.
  \chapter{Initial Work}
  Before starting with the implementation of multithreading, we evaluated the existing ELEXIR codebase to detect any potential issues. We used the \emph{Memcheck} tool within the \emph{Valgrind} program on Linux to detect memory leaks which we then fixed. Valgrind dynamically translates the machine code being executed and allows its tools to insert instrumentation before the code is actually executed. Memcheck tracks memory allocations and can warn of both memory leaks and invalid memory accesses.
  
  The reference-counting was changed to use reference-counting smart pointers rather than the manual incrementing and decrementing initially present. This helped fix a number of memory leaks. It was also modified to use atomic incrementing and decrementing of reference counts, and this was sufficient to ensure thread safety, since threads incrementing the reference count always own at least one reference to it through the explanation currently being processed. In order to verify that the changes made to the code did not break existing functionality, we added tests for the results of the algorithm on one of the domains. We also added unit tests for some of the new functionality.
  
  We considered several possible libraries for implementing multithreading. We chose the \emph{Boost Threads} library\cite{bib:threads}, which is part of the \emph{Boost} project, over the POSIX threading API for reasons of both convenience and portability; although the code has not yet been ported entirely to Windows, the added code was designed with this future port in mind.
  
  The executable compiled for evaluation runs takes as parameters the domain and problem files to be processed. In addition, command-line arguments allow specifying the number of explanations to allocate in each task, and the maximum number of worker threads to spawn. Except in Method 1, it will never spawn more worker threads than there are hardware threads available.
  
  The code for this project was developed using Boost version 1.46.1 on Ubuntu Linux 12.04 64-bit, and also tested and run using Boost version 1.41 on Scientific Linux 6.2. The machines used for evaluation are Sherlock, a 4-core Intel i5 desktop PC with 8GB of memory running the Ubuntu configuration, and Catzilla, a 16-core AMD Opteron multi-user server with 64GB of memory running the Scientific Linux configuration. The code was built using GCC under the -O3 optimisation level for evaluation.
  
  Boost 1.46.1 is the currently packaged version of Boost on Ubuntu 12.04. Unfortunately it suffers from an occasional deadlock issue due to locking order when a sleeping thread is interrupted by calling \texttt{boost::thread::interrupt()}. We encountered this issue during development which for certain build configurations and datasets would very reliably cause deadlocks. Interruption was being used to exit threads when processing was complete. Interruption causes an exception to be thrown in the interrupted thread; we replaced calls to \texttt{thread::interrupt()} with a task that throws this exception when invoked. The entry function of the worker thread then catches this exception and exits the thread.
  
  Each algorithm is implemented with a similar structure: a class encapsulates the algorithm, and is instantiated based on the selected algorithm for each run. This class has a method \texttt{explainObservations} which is called on the main thread, and creates the worker threads. Each of the threads runs a static method \texttt{thread\_fn} on the class, which interacts with a \texttt{WorkerData} structure containing the queues or other relevant information. The worker thread method is mostly independent of the nature of the work being done.
  Finally, tasks are represented by a \texttt{TaskData} structure which contains the expressions for the task, and any other relevant data. This structure has a method \texttt{execute} which is passed a pointer to the \texttt{WorkerData}, processes the expressions, and passes the results back into new tasks or to the main thread.
  
  The two problem domains used for evaluation are the XPERIENCE domain, which involves hands picking up and moving balls around and into cups, and is based on the XPERIENCE robotics project\cite{bib:xper}; and the Logistics domain, which involves trucks and airplanes transporting packages between cities, and is based on the Logistics domain in the First International Planning Competition\cite{bib:ipc}.
  
  During evaluation, each configuration was run 5 times, on both domains, on both machines, varying either the batch size or the maximum number of worker threads. We have focused on measuring the runtime for the core of the algorithm, ignoring in our timings the parsing of the input files, the probability computations, and output of the results.
  
  % TODO and the longest runtime discarded, to provide some insurance against runs where other processes are taking up a lot of processor time, particularly on Catzilla.
  
  Code listings will be provided to highlight certain details of implementation. In these listings, details such as i/o and runtime statistics gathering have been omitted for clarity.
  
  \chapter {Method 0 - Original Single-Threaded ELEXIR}
  
  The first method we include for comparison is a mostly unmodified single-threaded implementation, after the code had been made thread-safe and memory leaks addressed.
  
  \section {Implementation}
  
  On each new observation, the entire current list of explanations is looped over by the main thread, and a new list of explanations is generated. The batch size and thread count parameters are ignored.
  
  \begin{lstlisting}[language=C++]
  void explainObservations(Problem const* problem, list<Explanation*>* results)
  {
    // Initialise the set of explanations with a single empty explanation
    curExps.push_back( new Explanation() );
    // Loop over the observations
    for ( size_t i = 0; i < problem->obs.size(); ++ i ) {
      if ( curExps.empty() ) { return; } // Found no consistent explanations
      // Process all explanations on the main thread
      while ( !curExps.empty() ) {
        Explanation* lep = curExps.front();
        lep->extendExplanation( &problem->lexicon, &problem->obs[ lep->obsIndex ], &newExps );
        delete lep;
        curExps.pop_front();
      }
      curExps.splice(curExps.end(), newExps);
    }
    results->splice(results->end(), curExps);
  }
  \end{lstlisting}
  
  \section{Evaluation}
  
  The algorithm was evaluated on the XPERIENCE and Logistics domains, on the Sherlock and Catzilla machines. Since this method does not scale with the number of threads, we will report baseline performance numbers, shown in Table~\ref{tab:m0}.
  
  \begin{table}[!htbp]
  \begin{tabular}{c c c}
  Domain & Sherlock Runtime (seconds) & Catzilla Runtime (seconds)\tabularnewline
  \hline
  XPERIENCE Domain & 5.63 & 14.67\tabularnewline
  Logistics Domain & 1.36 & 3.59\tabularnewline
  \hline
  \end{tabular}
  \caption{Average Runtimes for Method 0}
  \label{tab:m0}
  \end{table}

  \chapter {Method 1 - One Thread Per Task}
  
  Before implementing the methods discussed in Section~\ref{sec:sched}, as a proof of concept for multithreading the algorithm, we implemented a scheduling algorithm which spawns a new thread for each task, one task per hardware thread up to the thread limit.
  
  \section {Implementation}
  
  On each new observation,  the main thread spawns a new thread for each task, equal to the number of hardware threads available or the thread limit. Each task processes its explanations and produces new resulting explanations. The main thread waits for all the spawned threads to complete, and collects their results, then spawns a new set of threads on the next observation. This implementation ignores the batch size parameter and partitions the explanations evenly between tasks.
  
  \begin{lstlisting}[language=C++]
  void explainObservations(Problem const* problem, list<Explanation*>* results)
  {
    // Initialise the set of explanations with a single empty explanation.
    curExps.push_back(new Explanation());
    // Loop over the observations
    for ( unsigned i = 0; i < problem->obs.size(); ++ i ) {
      if ( curExps.empty() ) { return; } // Found no consistent explanations
      // Distribute the explanations into one task per thread
      int i = 0;
      while ( !curExps.empty() ) {
        tasks[i].push_back(curExps.front());
        curExps.pop_front();
        i = (i + 1) % numThreads_;
      }
      // Launch the threads and wait for them to complete
      for (int i = 0; i < numThreads_; ++i) {
        threads.add_thread(new boost::thread(&thread_fn, problem, &tasks[i], &taskResults[i]));
      }
      threads.join_all();
      for (int i = 0; i < numThreads_; ++i) {
        curExps.splice(curExps.end(), taskResults[i]);
      }
    }
    results->splice(results->end(), curExps);
  }

  static void thread_fn(Problem const* problem, list<Explanation*>* exps, list<Explanation*>* out)
  {
    while ( !exps->empty() ) {
      Explanation* lep = exps->front();
      lep->extendExplanation(&problem->lexicon, &problem->obs[lep->obsIndex], out);
      delete lep;
      exps->pop_front();
    }
  }
  \end{lstlisting}

  \section{Evaluation}
  
  The algorithm was evaluated on the XPERIENCE and Logistics domains, on the Sherlock and Catzilla machines, with different thread count limits. For each set of runs, we plot the runtime in seconds and the relative throughput. If the algorithm scales perfectly, the throughput graph should be linear.
  
  \begin{figure}[!htbp]
  \centering{}
  \begin{minipage}[t]{0.49\columnwidth}
  \subfloat[Runtime]{
    \begin{centering}
    \includegraphics[scale=0.40]{{{images/threads-xper5-sherlock-1-1}}}
    \end{centering}
  }
  \end{minipage}
  \begin{minipage}[t]{0.49\columnwidth}
  \subfloat[Throughput]{
    \begin{centering}
    \includegraphics[scale=0.40]{{{images/threads-xper5-sherlock-1-2}}}
    \end{centering}
  }
  \end{minipage}
  \caption{Method 1 runtime with 1 to 4 threads on Sherlock, XPERIENCE}
  \label{fig:thread-s1x}
  \end{figure}
  
  \begin{figure}[!htbp]
  \centering{}
  \begin{minipage}[t]{0.49\columnwidth}
  \subfloat[Runtime]{
    \begin{centering}
    \includegraphics[scale=0.40]{{{images/threads-log3-sherlock-1-1}}}
    \end{centering}
  }
  \end{minipage}
  \begin{minipage}[t]{0.49\columnwidth}
  \subfloat[Throughput]{
    \begin{centering}
    \includegraphics[scale=0.40]{{{images/threads-log3-sherlock-1-2}}}
    \end{centering}
  }
  \end{minipage}
  \caption{Method 1 runtime with 1 to 4 threads on Sherlock, Logistics}
  \label{fig:thread-s1l}
  \end{figure}
  
  Despite the very simplistic approach to task scheduling taken in this approach, we achieve a good almost-linear speedup on Sherlock with this method, although the speedup visibly tails off at the 4th thread.
  
  \begin{figure}[!htbp]
  \centering{}
  \begin{minipage}[t]{0.49\columnwidth}
  \subfloat[Runtime]{
    \begin{centering}
    \includegraphics[scale=0.40]{{{images/threads-xper5-catzilla.inf.ed.ac.uk-1-1}}}
    \end{centering}
  }
  \end{minipage}
  \begin{minipage}[t]{0.49\columnwidth}
  \subfloat[Throughput]{
    \begin{centering}
    \includegraphics[scale=0.40]{{{images/threads-xper5-catzilla.inf.ed.ac.uk-1-2}}}
    \end{centering}
  }
  \end{minipage}
  \caption{Method 1 runtime with 1 to 16 threads on Catzilla, XPERIENCE}
  \label{fig:thread-c1x}
  \end{figure}
  
  \begin{figure}[!htbp]
  \centering{}
  \begin{minipage}[t]{0.49\columnwidth}
  \subfloat[Runtime]{
    \begin{centering}
    \includegraphics[scale=0.40]{{{images/threads-log3-catzilla.inf.ed.ac.uk-1-1}}}
    \end{centering}
  }
  \end{minipage}
  \begin{minipage}[t]{0.49\columnwidth}
  \subfloat[Throughput]{
    \begin{centering}
    \includegraphics[scale=0.40]{{{images/threads-log3-catzilla.inf.ed.ac.uk-1-2}}}
    \end{centering}
  }
  \end{minipage}
  \caption{Method 1 runtime with 1 to 16 threads on Catzilla, Logistics}
  \label{fig:thread-c1l}
  \end{figure}
  
  Catzilla provides a more challenging environment for evaluation. Since this is a multi-user system, the runtimes vary much more, but we can see in the XPERIENCE domain that the performance increases near-linearly up to 4 threads, then hardly increases after this. In the Logistics domain, the performance increases near-linearly up to 6 threads before it begins to level off.
  
  \chapter {Method 2 - One Queue Per Thread}
  
  In this section we discuss the implementation of the blocking queue method. This method uses one blocking queue per thread, guarded by a mutual exclusion lock.
  
  \section {Implementation}
  
  The main thread creates a queue for each worker thread, then spawns the threads. On each new observation, the main thread partitions the work evenly between the worker thread queues, and waits for them to finish processing them, finally collecting the new explanations.
  
  The implementation of the queue is based on boost::bounded\_buffer from the documentation of the Boost Circular Buffer library. Internally, it uses a bounded circular buffer guarded by a mutex. Worker threads trying to fetch work wait on a condition variable which is signalled when work is added to the buffer. The main difference between this method and the previous one is that the threads sleep between observations rather than exiting and being recreated.
  
  Completion is detected using a completion barrier as described in The Art of Multiprocessor Programming\cite{bib:aomp}. A count of active threads is maintained and atomically incremented and decremented by worker threads as they become busy or idle.
  
  This implementation does not take into account the batch size parameter, since the main thread is responsible for redistributing the work. The main thread creates as task for each worker thread and splits the explanation to be processed evenly between them.
  
  \begin{lstlisting}[language=C++]
  void explainObservations(Problem const* problem, list<Explanation*>* results)
  {
    for (int i = 0; i < numThreads_; ++i) {
      workerData[i].problem = problem;
    }
    // Initialise the set of explanations with a single empty explanation.
    curExps.push_back( new Explanation() );
    for (int i = 0; i < numThreads_; ++i) {
      threads.add_thread(new boost::thread(&thread_fn, &workerData[i]));
    }
    // Loop over the observations
    for ( unsigned i = 0; i < problem->obs.size(); ++ i ) {
      if ( curExps.empty() ) { return; } // Found no consistent explanations for the observations
      // Distribute the explanations into one task per thread
      int i = 0;
      while ( !curExps.empty() ) {
        taskData[i].exps.push_back(curExps.front());
        curExps.pop_front();
        i = (i + 1) % numThreads_;
      }
      for (int i = 0; i < numThreads_; ++i) {
        barrier.incActive();
        workerData[i].queue->push_front(&taskData[i]);
      }
      // Wait for them to complete
      while (!barrier.allInactive()) {
        boost::this_thread::sleep(boost::posix_time::millisec(1));
      }
      for (int i = 0; i < numThreads_; ++i) {
        curExps.splice(curExps.end(), taskData[i].results);
      }
    }
    results->splice(results->end(), curExps);
  }
  
  static void thread_fn(WorkerData* workerData)
  {
    Queue* queue = workerData->queue;
    TerminationBarrier* barrier = workerData->barrier;
    try {
      while (true) {
        TaskData* task;
        queue->pop_back(&task);
        task->execute(workerData);
        barrier->decActive();
      }
    } catch (boost::thread_interrupted& e) {
      // exit thread
    }
  }
  
  void Task::execute(WorkerData* workerData) {
    if (exit) { throw boost::thread_interrupted(); }
    Problem const* problem = workerData->problem;
    while ( !exps.empty() ) {
      Explanation* lep = exps.front();
      lep->extendExplanation(&problem->lexicon, &problem->obs[lep->obsIndex], &results);
      delete lep;
      exps.pop_front();
    }
  }
  \end{lstlisting}

  
  \section{Evaluation}
  
  The algorithm was again evaluated on the XPERIENCE and Logistics domains, on the Sherlock and Catzilla machines, with different thread count limits. For each set of runs, we plot the runtime in seconds and the relative throughput.
  
  \begin{figure}[!htbp]
  \centering{}
  \begin{minipage}[t]{0.49\columnwidth}
  \subfloat[Runtime]{
    \begin{centering}
    \includegraphics[scale=0.40]{{{images/threads-xper5-sherlock-2-1}}}
    \end{centering}
  }
  \end{minipage}
  \begin{minipage}[t]{0.49\columnwidth}
  \subfloat[Throughput]{
    \begin{centering}
    \includegraphics[scale=0.40]{{{images/threads-xper5-sherlock-2-2}}}
    \end{centering}
  }
  \end{minipage}
  \caption{Method 2 runtime, 1-4 threads on Sherlock, XPERIENCE}
  \label{fig:thread-s2x}
  \end{figure}
  
  \begin{figure}[!htbp]
  \centering{}
  \begin{minipage}[t]{0.49\columnwidth}
  \subfloat[Runtime]{
    \begin{centering}
    \includegraphics[scale=0.40]{{{images/threads-log3-sherlock-2-1}}}
    \end{centering}
  }
  \end{minipage}
  \begin{minipage}[t]{0.49\columnwidth}
  \subfloat[Throughput]{
    \begin{centering}
    \includegraphics[scale=0.40]{{{images/threads-log3-sherlock-2-2}}}
    \end{centering}
  }
  \end{minipage}
  \caption{Method 2 runtime, 1-4 threads on Sherlock, Logistics}
  \label{fig:thread-s2l}
  \end{figure}
  
  Both domains on Sherlock show very similar characteristics to the previous method, with almost linear speedup.
  
  \begin{figure}[!htbp]
  \centering{}
  \begin{minipage}[t]{0.49\columnwidth}
  \subfloat[Runtime]{
    \begin{centering}
    \includegraphics[scale=0.40]{{{images/threads-xper5-catzilla.inf.ed.ac.uk-2-1}}}
    \end{centering}
  }
  \end{minipage}
  \begin{minipage}[t]{0.49\columnwidth}
  \subfloat[Throughput]{
    \begin{centering}
    \includegraphics[scale=0.40]{{{images/threads-xper5-catzilla.inf.ed.ac.uk-2-2}}}
    \end{centering}
  }
  \end{minipage}
  \caption{Method 2 runtime, 1-16 threads on Catzilla, XPERIENCE}
  \label{fig:thread-c2x}
  \end{figure}
  
  \begin{figure}[!htbp]
  \centering{}
  \begin{minipage}[t]{0.49\columnwidth}
  \subfloat[Runtime]{
    \begin{centering}
    \includegraphics[scale=0.40]{{{images/threads-log3-catzilla.inf.ed.ac.uk-2-1}}}
    \end{centering}
  }
  \end{minipage}
  \begin{minipage}[t]{0.49\columnwidth}
  \subfloat[Throughput]{
    \begin{centering}
    \includegraphics[scale=0.40]{{{images/threads-log3-catzilla.inf.ed.ac.uk-2-2}}}
    \end{centering}
  }
  \end{minipage}
  \caption{Method 2 runtime, 1-16 threads on Catzilla, Logistics}
  \label{fig:thread-c2l}
  \end{figure}
  
  The runs on Catzilla are harder to analyse, but we can see that 5 threads seems the best number for the XPERIENCE domain, after which we have some erratic data points with the running time eventually rising again as the number of threads is increased. For the Logistics domain, performance seems to scale near-linearly up to 8 threads, after which it remains more or less constant.
    
  \chapter {Method 3 - Lock-Free Work-Stealing}
  
  In this section we discuss the implementation of the work-stealing queue method. This method uses one lock-free queue for each worker thread. Other threads can steal from this queue if they run out of work in their own queue.
  
  \section {Implementation}
  
  The implementation of this scheduler is based on\cite{bib:queue}, also described in The Art of Multiprocessor Programming\cite{bib:queue}, which describes a lock-free work-stealing queue where stealing threads steal a single task from the tail of the victim thread's queue, and threads otherwise insert and retrieve work from the head of their own queue. The queue can be resized safely if there is not enough space to insert work at the head of the queue. The original implementation described is appropriate for the language Java where memory is managed by garbage collection, and reads and writes to volatile variables are not reordered. In order to implement this in C++, the replacement of the queue's buffer when it is reallocated had to be guarded using \emph{hazard pointers}\cite{bib:hazard} and the ordering of reads and writes to variables accessed locklessly enforced using compiler-specific memory fence directives.
  
  In terms of performance, we expect this implementation to have low contention due to the lack of locking, and good load-balancing since idle threads will steal work from other threads. During initial testing, we found that performance would drop as the number of threads increased from 2 to 4 threads. Using cachegrind to produce a differential profile between a run with 1 thread and a run with 4 threads, we discovered that idle threads were spending a lot of time in random number generation, which is used when selecting a victim thread to steal from. The insertion of a short sleep after failed stealing attempts resolved this issue.
   
  \begin{lstlisting}[language=C++]
  void explainObservations(Problem const* problem, list<Explanation*>* results)
  {
    // Initialise the set of explanations with a single empty explanation
    TaskData initTask;
    initTask.exps.push_back(new Explanation());
    queues[0]->pushBottom( initTask );
    for (int i = 0; i < numThreads_; ++i) {
      workerData[i].problem = problem;
    }
    // Launch threads
    for (int i = 0; i < numThreads_; ++i) {
      threads.add_thread( new boost::thread(&thread_fn, &workerData[i]) );
    }
    // Wait for them to complete
    while (!barrier.allInactive()) {
      boost::this_thread::sleep(boost::posix_time::millisec(10));
    }
    // Force them to exit
    for (int i = 0; i < numThreads_; ++i) {
      TaskData exitTask;
      exitTask.exit = true;
      lexCore::writeBarrier();
      queues[i]->pushBottom(exitTask);
    }
    threads.join_all();
    for (int i = 0; i < numThreads_; ++i) {
      curExps.splice(curExps.end(), taskResults[i]);
    }
    results->splice(results->end(), curExps);
  }
  
  static void thread_fn(WorkerData* workerData)
  {
    VictimPicker* victimPicker = workerData->victimPicker;

    std::vector< WorkStealingQueue<TaskData>* >& queues = *workerData->queues;
    TerminationBarrier* barrier = workerData->barrier;
    boost::posix_time::millisec const sleepTime = boost::posix_time::millisec(1);

    TaskData task;
    try {
      while (true) {
        // Take work from own queue
        while (queues[workerData->workerID]->popBottom(&task)) {
          task.execute(workerData);
        }
        barrier->decActive();
        // Try stealing work
        while(true) {
          int victim = (*victimPicker)();
          if (!queues[victim]->isEmpty()) {
            barrier->incActive();
            if (queues[victim]->popTop(&task)) {
              task.execute(workerData);
              break;
            }
            barrier->decActive();
          }
          if (barrier->allInactive()) { return; }
          boost::this_thread::sleep(sleepTime);
        }
      }
    } catch (boost::thread_interrupted const& e) {
        // Exit thread
    }
  }

  void Task::execute(WorkerData* workerData)
  {
      if (exit) { throw boost::thread_interrupted(); }
      Problem const* problem = workerData->problem;
      while ( !exps.empty() ) {
        Explanation* lep = exps.front();
        if ((size_t)lep->obsIndex < problem->obs.size()) {
          list< Explanation* > results;
          lep->extendExplanation(&problem->lexicon, &problem->obs[lep->obsIndex], &results);
          delete lep;

          // Push results as new tasks
          while (!results.empty()) {
            for (int i = 0; i < workerData->maxBatch && !results.empty(); ++i) {
              TaskData newTask;
              newTask.exps.push_back(results.front());
              (*workerData->queues)[workerData->workerID]->pushBottom(newTask);
              results.pop_front();
            }
          }
        } else { // No more explanations, add this to the list of completed explanations
          workerData->results->push_back(lep);
        }

        exps.pop_front();
      }
    }
  \end{lstlisting}

  
  \section{Evaluation}
  
  \subsection{Batch Size Parameter}
  
  In order to estimate the best value for the batch size parameter, we ran the implementation with 4 threads on the XPERIENCE domain, varying the batch size.
  
  \begin{figure}[!htbp]
  \begin{centering}
  \includegraphics[width=0.8\textwidth]{{{images/batch-xper5-sherlock-3-1}}}
  \par\end{centering}
  \caption{Method 3 runtime, varying batch size on Sherlock, XPERIENCE}
  \label{fig:batch-3}
  \end{figure}
  
  We can see in Figure~\ref{fig:batch-3} that the batch size does not significantly affect the runtime for this algorithm. All further tests were performed with a batch size of 8.
  
  \subsection{Thread Count}
  
  The algorithm was again evaluated on the XPERIENCE and Logistics domains, on the Sherlock and Catzilla machines, with different thread count limits. For each set of runs, we plot the runtime in seconds and the relative throughput.
  
  \begin{figure}[!htbp]
  \centering{}
  \begin{minipage}[t]{0.49\columnwidth}
  \subfloat[Runtime]{
    \begin{centering}
    \includegraphics[scale=0.40]{{{images/threads-xper5-sherlock-3-1}}}
    \end{centering}
  }
  \end{minipage}
  \begin{minipage}[t]{0.49\columnwidth}
  \subfloat[Throughput]{
    \begin{centering}
    \includegraphics[scale=0.40]{{{images/threads-xper5-sherlock-3-2}}}
    \end{centering}
  }
  \end{minipage}
  \caption{Method 3 runtime, 1-4 threads on Sherlock, XPERIENCE}
  \label{fig:thread-s3x}
  \end{figure}
  
  \begin{figure}[!htbp]
  \centering{}
  \begin{minipage}[t]{0.49\columnwidth}
  \subfloat[Runtime]{
    \begin{centering}
    \includegraphics[scale=0.40]{{{images/threads-log3-sherlock-3-1}}}
    \end{centering}
  }
  \end{minipage}
  \begin{minipage}[t]{0.49\columnwidth}
  \subfloat[Throughput]{
    \begin{centering}
    \includegraphics[scale=0.40]{{{images/threads-log3-sherlock-3-2}}}
    \end{centering}
  }
  \end{minipage}
  \caption{Method 3 runtime, 1-4 threads on Sherlock, Logistics}
  \label{fig:thread-s3l}
  \end{figure}
  
  Runs on Sherlock again show a near-linear speedup with slight tailing off when we reach 4 threads.
  
  \begin{figure}[!htbp]
  \centering{}
  \begin{minipage}[t]{0.49\columnwidth}
  \subfloat[Runtime]{
    \begin{centering}
    \includegraphics[scale=0.40]{{{images/threads-xper5-catzilla.inf.ed.ac.uk-3-1}}}
    \end{centering}
  }
  \end{minipage}
  \begin{minipage}[t]{0.49\columnwidth}
  \subfloat[Throughput]{
    \begin{centering}
    \includegraphics[scale=0.40]{{{images/threads-xper5-catzilla.inf.ed.ac.uk-3-2}}}
    \end{centering}
  }
  \end{minipage}
  \caption{Method 3 runtime, 1-16 threads on Catzilla, XPERIENCE}
  \label{fig:thread-c3x}
  \end{figure}
  
  \begin{figure}[!htbp]
  \centering{}
  \begin{minipage}[t]{0.49\columnwidth}
  \subfloat[Runtime]{
    \begin{centering}
    \includegraphics[scale=0.40]{{{images/threads-log3-catzilla.inf.ed.ac.uk-3-1}}}
    \end{centering}
  }
  \end{minipage}
  \begin{minipage}[t]{0.49\columnwidth}
  \subfloat[Throughput]{
    \begin{centering}
    \includegraphics[scale=0.40]{{{images/threads-log3-catzilla.inf.ed.ac.uk-3-2}}}
    \end{centering}
  }
  \end{minipage}
  \caption{Method 3 runtime, 1-16 threads on Catzilla, Method 3, Logistics}
  \label{fig:thread-c3l}
  \end{figure}
  
  The runs on Catzilla show a sharp speedup at first, followed by a slow decline. The XPERIENCE domain peaks at around 5 threads, and the Logistics domain around 7 threads. 
  
  \chapter {Method 4 - Single Global Queue}
  
  In this section we discuss the global queue method. This method uses a single global queue of tasks, guarded by a mutual exclusion lock.
  
  \section {Implementation}
  
  The main thread adds a single empty explanation to the global queue, then spawns a worker thread for each hardware thread available. These threads all attempt to lock the queue and extract work from it. If they cannot lock the queue, they sleep until the lock becomes available. If they lock the queue and find it empty, they sleep for a fixed period of time, then try again. Once they complete the task, they lock the queue to put all the newly generated work on the queue, then resume trying to pull a task from the head of the queue.
  
  \begin{lstlisting}[language=C++]
  void explainObservations(Problem const* problem, list<Explanation*>* results)
  {
    for (int i = 0; i < numThreads_; ++i) {
      workerData[i].problem = problem;
    }

    // Initialise the set of explanations with a single empty explanation
    {
      TaskData* newTask = new TaskData();
      newTask->exps.push_back(new Explanation());
      boost::mutex::scoped_lock lock(mutex);
      queue.push_back(newTask);
    }

    // Create worker threads and wait for them to complete
    for (int i = 0; i < numThreads_; ++i) {
      threads.add_thread( new boost::thread(&thread_fn, &workerData[i]) );
    }
    while (!barrier.allInactive()) {
      boost::this_thread::sleep(boost::posix_time::millisec(10));
    }
    threads.interrupt_all();
    threads.join_all();

    for (int i = 0; i < numThreads_; ++i) {
      curExps.splice(curExps.end(), taskResults[i]);
    }
    if ( curExps.empty() ) { return; } // Found no consistent explanations for the observations
    results->splice(results->end(), curExps);
  }

  static void thread_fn(WorkerData* workerData) {
    TerminationBarrier* barrier = workerData->barrier;
    boost::mutex* mutex = workerData->mutex;
    Queue* queue = workerData->queue;

    boost::posix_time::millisec const sleepTime = boost::posix_time::millisec(10);
    
    try {
      while (true) {
        TaskData* task = NULL;
        {
          boost::mutex::scoped_lock lock(*mutex);
          if (!queue->empty()) {
            task = queue->front();
            queue->pop_front();
          }
        }

        // If succeeded, execute, otherwise sleep
        if (task) {
          task->execute(workerData);
          delete task;
        } else {
          barrier->decActive();
          if (barrier->allInactive()) { return; }
          boost::this_thread::sleep(sleepTime);
          barrier->incActive();
        }
      }
    } catch (boost::thread_interrupted const& e) {
      // exit thread
    }
  }
  
  void execute(WorkerData* workerData)
  {
    if (exit) { throw boost::thread_interrupted(); }
    Problem const* problem = workerData->problem;
    std::list<Explanation*> results;
    while ( !exps.empty() ) {
      Explanation* lep = exps.front();
      exps.pop_front();
      if ((size_t)lep->obsIndex < problem->obs.size()) {
        lep->extendExplanation(&problem->lexicon, &problem->obs[lep->obsIndex], &results);
        delete lep;
      } else {
        // No more explanations, add this to the list of completed explanations
        workerData->results->push_back(lep);
      }
    }

    // Push new expression back onto queue as tasks
    boost::mutex::scoped_lock lock(*workerData->mutex);
    while (!results.empty()) {
      TaskData* newTask = new TaskData;
      for (int i = 0; i < workerData->maxBatch && !results.empty(); ++i) {
        newTask->exps.push_back(results.front());
        results.pop_front();
      }
      workerData->queue->push_back(newTask);
    }
  }
  \end{lstlisting}

  
  \section{Evaluation}
  
  \subsection{Batch Size Parameter}
  
  In order to estimate the best value for the batch size parameter, we again ran the implementation with 4 threads on the XPERIENCE domain, varying the batch size.
  
  \begin{figure}[!htbp]
  \begin{centering}
  \includegraphics[width=0.8\textwidth]{images/batch-xper5-sherlock-4-1}
  \par\end{centering}
  \caption{Runtime, varying batch size on Sherlock, Method 3, XPERIENCE}
  \label{fig:batch-4}
  \end{figure}
  
  We can see in Figure~\ref{fig:batch-4} that the batch size parameter does have an effect on the runtime of this algorithm, suggesting that contention for the lock is worse with small tasks leading for more frequent insertions and removals on the global queue. In this test the runtime is similar from 8 explanations per batch upward, so all further tests were performed with a batch size of 8.
  
  \subsection{Thread Count}
  
  The algorithm was again evaluated on the XPERIENCE and Logistics domains, on the Sherlock and Catzilla machines, with different thread count limits. For each set of runs, we plot the runtime in seconds and the relative throughput.
  
  \begin{figure}[!htbp]
  \centering{}
  \begin{minipage}[t]{0.49\columnwidth}
  \subfloat[Runtime]{
    \begin{centering}
    \includegraphics[scale=0.40]{{{images/threads-xper5-sherlock-4-1}}}
    \end{centering}
  }
  \end{minipage}
  \begin{minipage}[t]{0.49\columnwidth}
  \subfloat[Throughput]{
    \begin{centering}
    \includegraphics[scale=0.40]{{{images/threads-xper5-sherlock-4-2}}}
    \end{centering}
  }
  \end{minipage}
  \caption{Method 4 runtime, 1-4 threads on Sherlock, XPERIENCE}
  \label{fig:thread-s4x}
  \end{figure}
  
  \begin{figure}[!htbp]
  \centering{}
  \begin{minipage}[t]{0.49\columnwidth}
  \subfloat[Runtime]{
    \begin{centering}
    \includegraphics[scale=0.40]{{{images/threads-log3-sherlock-4-1}}}
    \end{centering}
  }
  \end{minipage}
  \begin{minipage}[t]{0.49\columnwidth}
  \subfloat[Throughput]{
    \begin{centering}
    \includegraphics[scale=0.40]{{{images/threads-log3-sherlock-4-2}}}
    \end{centering}
  }
  \end{minipage}
  \caption{Method 4 runtime, 1-4 threads on Sherlock, Logistics}
  \label{fig:thread-s4l}
  \end{figure}
  
  The runs on Sherlock again show a near-linear speedup.
  
  \begin{figure}[!htbp]
  \centering{}
  \begin{minipage}[t]{0.49\columnwidth}
  \subfloat[Runtime]{
    \begin{centering}
    \includegraphics[scale=0.40]{{{images/threads-xper5-catzilla.inf.ed.ac.uk-4-1}}}
    \end{centering}
  }
  \end{minipage}
  \begin{minipage}[t]{0.49\columnwidth}
  \subfloat[Throughput]{
    \begin{centering}
    \includegraphics[scale=0.40]{{{images/threads-xper5-catzilla.inf.ed.ac.uk-4-2}}}
    \end{centering}
  }
  \end{minipage}
  \caption{Method 4 runtime, 1-16 threads on Catzilla, XPERIENCE}
  \label{fig:thread-c4x}
  \end{figure}
  
  \begin{figure}[!htbp]
  \centering{}
  \begin{minipage}[t]{0.49\columnwidth}
  \subfloat[Runtime]{
    \begin{centering}
    \includegraphics[scale=0.40]{{{images/threads-log3-catzilla.inf.ed.ac.uk-4-1}}}
    \end{centering}
  }
  \end{minipage}
  \begin{minipage}[t]{0.49\columnwidth}
  \subfloat[Throughput]{
    \begin{centering}
    \includegraphics[scale=0.40]{{{images/threads-log3-catzilla.inf.ed.ac.uk-4-2}}}
    \end{centering}
  }
  \end{minipage}
  \caption{Method 4 runtime, 1-16 threads on Catzilla, Logistics}
  \label{fig:thread-c4l}
  \end{figure}
  
  The runs on Catzilla show a linear speedup at first, followed by a peak and tailing off. Both domains peak at around 8 threads.
  
  % \input{sections/04_design}
   
  % \input{sections/05_implementation}
  
  % Analysis: results and their critical analysis should be reported, whether the results conform to expectations or otherwise and how they compare with other related work.
  
  \chapter{Evaluation}

TODO here compare the different approaches, have graphs with results from all algorithms

TODO describe choice of metrics (or do this earlier?)

TODO describe the domains used (or do this earlier?)

TODO discuss hyperthreading and how it affects scaling of performance with number of threads (mention this in background too?)

TODO compare the speedup we get to a perfect $T_1$ to $T_\infty$ speedup
  % Conclusion:
  \chapter{Conclusion}

TODO draw conclusions from the results - why is work stealing so much better for this problem?

TODO future work: exploring only the most likely explanations given the observations so far

TODO future work: improving the memory allocation behaviour of the algorithm and testing different memory allocators better suited to multi-threaded code

TODO future work: paralellising the probability computation (would be automatic, and probably have better memory locality, if we did it incrementally)

TODO future work: incremental version, producing explanations each observations

TODO future work: discarding observations of a certain age if unused?

  % Appendix
  \appendix

  % \input{sections/appendix/A_evaluation.tex}

  % Bibliography
  % \bibliography{sections/misc/z_bib.tex}{}
  % \bibliographystyle{unsrt}
  \begin{thebibliography}{100}
  \bibitem{bib:netsec} C. W. Geib and R. P. Goldman, ``Plan Recognition in Intrusion Detection Systems'', in \emph{DARPA Information Survivability Conference and Exposition (DISCEX)}, 2001.
  \bibitem{bib:jast} M. E. Foster, T. By, M. Rickert, and A. Knoll, ``Human-Robot Dialogue for Joint Construction Tasks'', \emph{ICMI}, 2006.
  \bibitem{bib:assistive} M. E. Pollack, ``Intelligent technology for an aging population: The use of AI to assist elders with cognitive impairment'', \emph{AI Magazine}, vol. 26, no. 2, pp. 9-24, 2005.
  \bibitem{bib:ccg} M. Steedman, ``The Syntactic Process'', MIT Press, 2000.
  \bibitem{bib:ccg2} B. Hoffman ``Integrating Free Word Order Syntax and Information Structure'', in \emph{Proceedings of the seventh conference on European chapter of the Association for Computational Linguistics (EACL'95)}, pp. 245-252, 1995.
  \bibitem{bib:elexir} C. W. Geib, ``Delaying Commitment in Plan Recognition Using Combinatory Categorial Grammars'', in \emph{Proceedings of the International Joint Conference on Artificial Intelligence (IJCAI)}, pp. 1702-1707, 2009.
  \bibitem{bib:aomp} M. Herlihy and N. Shavit, ``The Art of Multiprocessor Programming, Revised First Edition'', Morgan Kaufmann, 2012.
  \bibitem{bib:hmm} H. H. Bui, S. Venkatesh, and G. West, ``Policy recognition in the Abstract Hidden Markov Model'', \emph{Journal of Artificial Intelligence Research}, vol. 17, pp. 451-499.
  \bibitem{bib:graph} D. Avrahami-Zilberbrand and G. A. Kaminka, ``Fast and Complete Symbolic Plan Recognition'', in \emph{Proceedings IJ-CAI}, 2005.
  \bibitem{bib:search} G. R. Andrews, ``Foundations of Multithreaded, Parallel, and Distributed
Programming'', Addison-Wesley, 2000.
  \bibitem{bib:parsing} D. Grune, ``Parsing Techniques: A Practical Guide'', Springer, 2008.
  \bibitem{bib:ccgpar} S. Clark and J. R. Curran, ``Log-Linear Models for Wide-Coverage CCG Parsing'', in \emph{Proceedings of the 2003 Conference on Empirical Methods in Natural
Language Processing}, pp. 97-104, 2003.
 \bibitem{bib:workstealing} R. D. Blumofe and C.E. Leiserson, ``Scheduling Multithreaded Computations by Work Stealing'', in \emph{Journal of the ACM (JACM)}, vol. 46, issue 5 pp. 720-748, 1999.
 \bibitem{bib:queue} D. Chase and Y. Lev, ``Dynamic Circular Work-Stealing Deque'', in \emph{Proceedings of the Seventeenth Annual ACM Symposium on Parallelism in Algorithms and Architectures (SPAA'05)}, pp. 21-28, 2005.
 \bibitem{bib:threads} A. Williams, ``What's New in Boost Threads?'', \emph{Dr. Dobb's Journal}, November 2008 issue.
 \bibitem{bib:xper} http://www.xperience.org/.
 \bibitem{bib:ipc} D. Long and M. Fox, ``The 3rd international Planning Competition: Results and Analysis'', \emph{Journal of Artificial Intelligence Research}, vol. 20, pp. 1-59, 2003.
 \bibitem{bib:hazard} M. Michael, ``Hazard Pointers: Safe Memory Reclamation for Lock-Free Objects'', \emph{IEEE Transactions on Parallel and Distributed Systems}, vol. 15 issue 6, pp. 491-504, 2004.
  \end{thebibliography}

\end{document}
