\chapter{Conclusion}

Due to the widespread usage of database systems throughout the modern world it
is essential that there are usable interfaces for using and exploring the data
inside them. These interfaces must be generic and fit many kinds of data in
order to be reused cheaply and without waste for many different databases. By
opening up these databases to a wider audience we enable users with expertise
in fields that are able to use this data without requiring that they learn the
intricacies of databases.

In this project I have designed and built what I believe to be one possible
solution to this problem. We allow far more users to explore the contained data
by presenting the data contained in the records and tables of a database to the
user in the familiar interface paradigm of files and directories.

Being able to explore and edit the data easily is just the first step in
improving the accessibility to all users. The next step is allowing the user to
perform more complex queries of this information. To accomplish this I have
built a new application that allows users to create queries using a graph based
interface. Users unequivocally preferred using this interface compared to using
\ac{SQL} even if they knew the latter technology well.

I believe that the tools above, when combined, form a system that is a solution
to the original problem of general database accessibility and visualisation for
users of all skill levels.
