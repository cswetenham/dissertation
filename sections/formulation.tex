\chapter{Formulation of the Problem \& Proposed Solution}

\section{Current Solutions}

In order to make databases easier to use for everyone we must first work out why they are difficult to use or how current solutions fall short. We see that most current database frontends are either custom built or ugly general purpose table styled editors that make any meaningful edits hard to comprehend.

The custom built solutions usually don't look anything like a database and on the contrary completely hide the fact that there is a database powering them. Web applications such as Facebook are classic examples of this. Behind the scenes there will be databases holding and retrieving information but there will never be a single mention of a select, join, table or view on the front end for the user to contend with. The problem with taking this solution to the database accessibility problem is how specific and bespoke it is. We are unable to generalise this beyond very simple cases due to the varying scales and types of data stored in databases.

The second solution to the problem currently involves displaying the information in the database in its root form of tables, rows and views. This information is then logically displayed in lists and tables in the user interface for manipulation and querying. This solution shows the core ``data structures'' of the database and is not suited for novice users as it requires knowledge of these concepts to use.

\section{Proposed Solution}

We need to find a good compromise between these two extremes: user-friendly but bespoke and hard-to-use but powerful. In order to find this compromise I am going to present the user with a new type of view of the database: a file and directory tree. This decision was made because users are familiar with manipulating files and folders already and they fit the shallow hierarchical structure of a database. It also allows the user to not just manipulate the files and directories but also edit their contents. If we map from records to files then editing a file can correspond to updating the record.