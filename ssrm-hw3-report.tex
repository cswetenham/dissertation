%% LyX 2.0.0 created this file.  For more info, see http://www.lyx.org/.
%% Do not edit unless you really know what you are doing.
\documentclass[english]{article}
\usepackage{beramono}
\usepackage[T1]{fontenc}
\usepackage[latin9]{inputenc}
\usepackage{listings}
\usepackage{array}
\usepackage{float}
\usepackage{graphicx}
\usepackage[all]{xy}

\makeatletter

%%%%%%%%%%%%%%%%%%%%%%%%%%%%%% LyX specific LaTeX commands.
%% Because html converters don't know tabularnewline
\providecommand{\tabularnewline}{\\}

%%%%%%%%%%%%%%%%%%%%%%%%%%%%%% User specified LaTeX commands.
\usepackage[usenames,dvipsnames]{color}
\xyoption{frame}

\@ifundefined{showcaptionsetup}{}{%
 \PassOptionsToPackage{caption=false}{subfig}}
\usepackage{subfig}
\makeatother

\usepackage{babel}
\begin{document}

\title{SSRM Homework 3}


\author{Chris Swetenham (s1149322)}


\date{March 22, 2012}

\maketitle

\section{Automated Driving}


\subsection*{a)}

We will consider the automated driving problem as a game at a single
instant in time. The state will consist of:
\begin{itemize}
\item $d$ -- The distance between the robot driver and the car in front
\item $v_{r}$ -- The velocity of the car on the right
\item $\theta_{r}$ -- The heading of the car on the right
\end{itemize}
The payoffs for the robot driver will be -1000 in case of collision,
-1 if we choose to Brake, 0 otherwise. We will not use an explicit
utility for the driver on the right but they are trying to turn into
the lane and avoid collision.

The actions for the robot driver will be: Brake, or Nothing (keeping
current speed). The actions for the driver on the right will be Accelerate,
Turn, Signal, or Nothing.

The driver in front will be treated as a random process rather than
an agent, since a driver will not usually be expected to respond to
the actions of the cars behind it, only in front of it and around
it.

The influence diagram is shown in Figure \ref{fig:1a}. Since we are
only considering a single instant in time, we are ignoring a lot of
the variables which take an effect over time, and only considering
the several possible channels of information from the Human driver
to the right, as well as our distance to the driver in front, all
of which we observe with some noise. The Human makes some choice and
we react to it and to the driver in front by choosing whether to Brake. 

\begin{figure}[H]
\centering{}$\xymatrix{ & *++[F]{Human}\ar[d]\\
*++{d}\ar[dr]\ar[ur]\save"2,1"*\frm<8pt>{-}\restore & *++{v,\theta,s}\ar[d]\save"2,2"*\frm<8pt>{-}\restore\\
 & *++[F]{Robot}\ar[d]\\
 & *++[F=]{Outcome}
}
$\caption{Influence Diagram for Automated Driving}
\label{fig:1a}
\end{figure}



\subsection*{b)}

We will assign the following strategy to the human driver on the right:
if it observes $d>d_{h}$ for some $d_{h}$ it will signal with probability
$\frac{1}{2}$ and accelerate and turn with probability $\frac{1}{2}$.
The car in front will continue to move at random speed and this is
observed through the noisy variable $d$.

As a decision tree formulation, since the Human's action depends on
$d$, we want to have the node for $d$ at the root, then the nodes
for the human's action, then the noisy observation, then finally the
Robot's action and the outcome. We will show this only for the Turn
action to make the diagram smaller. This is shown in Figure \ref{fig:1b}.
If the robot observes a noisy signal and decides to brake or do nothing.

\begin{figure}[H]
\centering{}$\xymatrix{ & *++{d}\ar[d]\save"1,2"*\frm<8pt>{-}\restore\\
 & *++[F]{Human}\ar[d]^{\theta}\\
 & *++{\theta}\ar[d]\save"3,2"*\frm<8pt>{-}\restore\\
 & *++[F]{Robot}\ar[dl]_{brake}\ar[dr]^{nothing}\\
*++[F=]{Safe} &  & *++[F=]{Crash}
}
$\caption{Decision Diagram for Automated Driving}
\label{fig:1b}
\end{figure}



\subsection*{c)}

Figure \ref{fig:1b} shows an instance of the problem where the Human
has decided. Since the Human has decided to turn, if the Robot decides
to do nothing it will crash. The robot observes a noisy value of $\theta+\varepsilon$
. The only type of instant-by-instant policy which would make sense
is one which uses a threshold on the observed value, and if the direction
is more than a certain threshold away from straight ahead, we will
Brake. If $\theta_{k}$ is a threshold on the real heading that serves
well to discriminate the actions Turn and Nothing by the driver, the
best policy for the robot will include a safety margin due to the
noise, and so brake if $\theta+\varepsilon>\theta_{k}-c\sigma_{\varepsilon}$.
By varying $c$, we can control the relative probabilities of false
positives or false negatives. The correct value of this threshold
will be one that balances the negative reward of Braking and the negative
utility of a Crash. \clearpage{}


\section{Rock Paper Scissors}

We give the matrix game and influence diagram for Rock, Paper, Scissors
in Table \ref{tab:2a-1} and Figure \ref{fig:2a-1}. Each player makes
a move without knowing the other player's move.

\begin{center}
\begin{table}[H]
\begin{centering}
\begin{tabular}{c|r>{\centering}p{0.01\textwidth}lr>{\centering}p{0.01\textwidth}lr>{\centering}p{0.01\textwidth}l}
P1\textbackslash{}\textsuperscript{P2} & \multicolumn{3}{c}{Rock} & \multicolumn{3}{c}{Paper} & \multicolumn{3}{c}{Scissors}\tabularnewline
\hline 
Rock & 0 & , & 0 & -1 & , & 1 & 1 & , & -1\tabularnewline
Paper & 1 & , & -1 & 0 & , & 0 & -1 & , & 1\tabularnewline
Scissors & -1 & , & 1 & 1 & , & -1 & 0 & , & 0\tabularnewline
\end{tabular}
\par\end{centering}

\caption{Matrix Game for Rock, Paper, Scissors}
\label{tab:2a-1}
\end{table}

\par\end{center}

\begin{center}
\begin{figure}[H]
\begin{centering}
$\xymatrix{*++[F]{Opponent\; Move}\ar[dr] &  & *++[F]{Player\; Move}\ar[dl]\\
 & *++[F=]{Outcome}
}
$
\par\end{centering}

\caption{Influence Diagram for Rock, Paper, Scissors}
\label{fig:2a-1}
\end{figure}

\par\end{center}

We now fix the strategy of the opponent, shown in Figure \ref{fig:2a-2},
and create the decision diagram, shown in Figure \ref{fig:2a-3}.
Due to space constraints we don't expand the random node for the opponent
move, but the opponent move nodes in fact should branch to all 3 possible
moves with probabilities weighting each edge and the outcomes can
be assigned a reward $[-1,0,1]$.

\begin{center}
\begin{figure}[H]
\begin{centering}
$\xymatrix{*++{Opponent\; Move}\ar[dr]\save"1,1"*\frm<8pt>{-}\restore &  & *++[F]{Player\; Move}\ar[dl]\\
 & *++[F=]{Outcome}
}
$
\par\end{centering}

\caption{Influence Diagram for Rock, Paper, Scissors after setting opponent
strategy}
\label{fig:2a-2}
\end{figure}

\par\end{center}

\begin{figure}[H]
\begin{centering}
$\xymatrix{ & *++[F]{Player\; Move}\ar[dl]^{r}\ar[d]^{p}\ar[dr]_{s}\\*
++{Opponent\; Move}\ar[d]\save"2,1"*\frm<8pt>{-}\restore & *++{Opponent\; Move}\ar[d]\save"2,2"*\frm<8pt>{-}\restore & *++{Opponent\; Move}\ar[d]\save"2,3"*\frm<8pt>{-}\restore\\*
++[F=]{Outcome} & *++[F=]{Outcome} & *++[F=]{Outcome}
}
$
\par\end{centering}

\caption{Influence Diagram for Rock, Paper, Scissors}
\label{fig:2a-3}
\end{figure}


The optimal strategy given the opponent's strategy is a pure strategy:
we simply pick the move with the highest expected return. Here the
opponent's strategy is a mixed one, with $P(Rock)=0.25;\; P(Paper)=0.4;\; P(Scissors)=0.35$.

$Q(Rock)=0\times P(Rock)+-1\times P(Paper)+1\times P(Scissors)=0.05$

$Q(Paper)=1\times P(Rock)+0\times P(Paper)+-1\times P(Scissors)=-0.1$

$Q(Scissors)=-1\times P(Rock)+1\times P(Paper)+0\times P(Scissors)=0.15$

Our optimal strategy is to pick $Scissors$.

We now perform fictitious plays for 5000 rounds with the above opponent
strategy. We estimate the opponent's strategy based on observed moves
in previous rounds. We graph the average reward over time, shown in
Figure \ref{fig:2a-4}. The value converges to 0.15 as expected.

\medskip{}


\lstinputlisting[basicstyle={\scriptsize\ttfamily},breaklines=true,caption={simGame.m},captionpos=b,commentstyle={\color{OliveGreen}},frame=tb,language=Matlab]{simGame.m}
\begin{figure}[H]
\begin{centering}
\includegraphics[width=0.8\textwidth]{q2a}
\par\end{centering}

\caption{Average Reward}
\label{fig:2a-4}
\end{figure}


For the next part I am going to assume the payoffs of +1 go to the
player and not the opponent, otherwise we do not have a very interesting
problem. The results of running our algorithm for 1000 rounds with
3 different values of p, the probability the opponent takes action
1, are shown in Figure \ref{fig:2b}.

\begin{center}
\begin{table}[H]
\begin{centering}
\begin{tabular}{c|r>{\centering}p{0.01\textwidth}lr>{\centering}p{0.01\textwidth}l}
P1\textbackslash{}\textsuperscript{P2} & \multicolumn{3}{c}{1} & \multicolumn{3}{c}{2}\tabularnewline
\hline 
1 & 1 & , & 0 & 0 & , & 0\tabularnewline
2 & 0 & , & 0 & 1 & , & 0\tabularnewline
\end{tabular}
\par\end{centering}

\caption{Matrix for second game}
\label{tab:2b-1}
\end{table}

\par\end{center}

\begin{figure}[H]
\begin{centering}
\begin{minipage}[t]{0.33\columnwidth}%
\begin{center}
\subfloat[$p=0.5$]{\includegraphics[width=1\textwidth]{q2b-1}}
\par\end{center}%
\end{minipage}%
\begin{minipage}[t]{0.33\columnwidth}%
\begin{center}
\subfloat[$p=0.25$]{\includegraphics[width=1\textwidth]{q2b-2}}
\par\end{center}%
\end{minipage}%
\begin{minipage}[t]{0.33\columnwidth}%
\begin{center}
\subfloat[$p=1.0$]{\includegraphics[width=1\textwidth]{q2b-3}}
\par\end{center}%
\end{minipage}
\par\end{centering}

\caption{Average Rewards}
\label{fig:2b}
\end{figure}


\medskip{}


\lstinputlisting[basicstyle={\scriptsize\ttfamily},breaklines=true,caption={q2.m},captionpos=b,commentstyle={\color{OliveGreen}},frame=tb,language=Matlab]{q2.m}\clearpage{}


\section{Rumour Spreading}


\subsection*{a)}

There is some ambiguity in the question - we could keep N constant
and allow z to vary with our initial x and y, or keep z to 0 initially
and allow N to vary according to x, y, z. I choose the former interpretation,
since it gives us trajectories in the state space of the same system.
We keep N constant and sample initial values of x and y such that
$x+y<N+1$. As we see in Figure \ref{fig:q3a}, depending on the initial
values the state can reach an equilibrium where 25 people will never
hear the rumour, and the final number of people ignorant of the rumour
does not easily reach all the way to 0.

\medskip{}


\lstinputlisting[basicstyle={\scriptsize\ttfamily},breaklines=true,caption={rumourOde.m},captionpos=b,commentstyle={\color{OliveGreen}},frame=tb,language=Matlab]{rumourOde.m}

\begin{figure}[H]
\centering{}\includegraphics[width=0.8\textwidth]{q3a}\caption{State-space trajectories}
\label{fig:q3a}
\end{figure}



\subsection*{b)}

Starting from initial conditions $y=1$, $x=N=75$ with $\mu=0.75$,
we run ode45 for 3 time units with a 0.001 unit timestep. The results
are shown in Figure \ref{fig:q3b}. The starting point is at the right
end, and the system comes to rest in the state at the left.

\begin{figure}[H]


\centering{}\includegraphics[width=0.8\textwidth]{q3b}\caption{State-space trajectory}
\label{fig:q3b}
\end{figure}



\subsection*{c)}

We vary the parameter $\mu$ and observe the effect on the system
dynamics. In Figure \ref{fig:q3c-1} we see the result of setting
$\mu=0.03$. The state trajectory is the same, but much slower, and
the state does not reach equilibrium in 3 time units. In Figure \ref{fig:q3c-2}
we see the result of setting $\mu=5.00$. The state trajectory is
the same, but convergence happens faster, and the curve is a little
less smooth as a result since fewer timesteps were spent in each section
of the curve before reaching an equilibrium state.

\begin{figure}[H]
\centering{}%
\begin{minipage}[t]{0.5\columnwidth}%
\subfloat[$\mu=0.03$]{

\begin{centering}
\includegraphics[width=1\textwidth]{q3c-03}\label{fig:q3c-1}
\par\end{centering}

}%
\end{minipage}%
\begin{minipage}[t]{0.5\columnwidth}%
\subfloat[$\mu=5.00$]{\centering{}\includegraphics[width=1\textwidth]{q3c-500}\label{fig:q3c-2}}%
\end{minipage}\caption{Effect of varying $\mu$}
\label{fig:q3c}
\end{figure}


\medskip{}


\lstinputlisting[basicstyle={\scriptsize\ttfamily},breaklines=true,caption={q3.m},captionpos=b,commentstyle={\color{OliveGreen}},frame=tb,language=Matlab]{q3.m}

\clearpage{}


\section*{Appendix A - Additional Code}

\medskip{}


\lstinputlisting[basicstyle={\scriptsize\ttfamily},breaklines=true,caption={main.m},captionpos=b,commentstyle={\color{OliveGreen}},frame=tb,language=Matlab]{main.m}

\medskip{}
\lstinputlisting[basicstyle={\scriptsize\ttfamily},breaklines=true,caption={writeFigureEPS.m},captionpos=b,commentstyle={\color{OliveGreen}},frame=tb,language=Matlab]{writeFigureEPS.m}
\end{document}
